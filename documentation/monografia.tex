\documentclass[a4paper,12pt,openright,oneside]{book}
\usepackage[portuges, brazil]{babel}   
\usepackage[figuresright]{rotating}
\usepackage{amsthm}
\usepackage{graphics}
\usepackage{amssymb}
\usepackage{graphicx}
\usepackage{fancybox}
\usepackage{amsmath}
\usepackage{picinpar}
\usepackage{colortbl}
\usepackage{wasysym}
\usepackage{txfonts}
\usepackage{pb-diagram}
\usepackage{relsize}
\usepackage{tikz}
	\usetikzlibrary{calc}
	\usetikzlibrary{datavisualization}
	\usetikzlibrary{positioning}
	\usetikzlibrary{mindmap}
	\usetikzlibrary{snakes}
	\usetikzlibrary{shapes}
	\usetikzlibrary{decorations.pathreplacing}
	\usetikzlibrary{spy}
	\usetikzlibrary{backgrounds}
	\usetikzlibrary{patterns}
\usepackage{pgfplots}
\usepackage{pgfplotstable}
	\pgfplotsset{compat=newest}
	\usepgfplotslibrary{units}
\usepackage{subfigure}
\usepackage{algorithm}
\usepackage{algorithmic}
\usepackage{verbatim}
\usepackage{wrapfig}
\usepackage{array}
\usepackage{calc}
\usepackage[T1]{fontenc}
\usepackage{times}
\usepackage{indentfirst}        % indenta primeiro par�grafo
\usepackage{fancyhdr}
\usepackage{pifont}
\usepackage{textcomp}      % \texttrademark
\usepackage{url}  
\usepackage{multirow}  
\usepackage[numbers]{natbib}
\usepackage{notoccite}
\usepackage{setspace}
\usepackage{array}
\usepackage{helvet}
\usepackage{csvsimple}
\renewcommand{\familydefault}{\sfdefault}
\headheight 16pt
\setlength{\topmargin}{-15pt} % extra vert. space + at the top of header: 23pt
\setlength{\oddsidemargin}{0pt} % extra spc added at the left of odd page: 0pt
\setlength{\evensidemargin}{-12pt} % ext. spc added at the left of even pg: 59pt
\setlength{\textheight}{638pt} % height of the body: 592pt
\setlength{\textwidth}{483pt} % width of the body: 470pt
\pagestyle{fancyplain}
\renewcommand{\chaptermark}[1]{\markboth{#1}{}}
\renewcommand{\sectionmark}[1]{\markright{\thesection\ #1}}
\lhead[\fancyplain{}{\bfseries\thepage}]{\fancyplain{}{\bfseries\rightmark}}
\rhead[\fancyplain{}{\bfseries\leftmark}]{\fancyplain{}{\bfseries\thepage}}
\cfoot[\fancyplain{\bfseries\thepage}{}]{\fancyplain{\bfseries\thepage}{}}
\newenvironment{myenv}[1]
  {\begin{spacing}{#1}}
  {\end{spacing}}
%%%%%%%%%%%%%%%%%%%%%%%%%%%%%%%%%%%%%%%%%%%%%%%%%%%%%%%%%%%%%%%%%%%%%%%%%
%%%%%%%%%%%%%%%%%%%%%%%%%%%%%%%%%%%%%%%%%%%%%%%%%%%%%%%%%%%%%%%%%%%%%%%%%
%% Inicia o texto
\begin{document}
%% Abrevia figuras e tabelas

%\def\figurename{Fig.}
%\def\tablename{Tab.}
%% Paginas iniciais (sem numerar)
%\frontmatter
%%
%% **********CAPA 
%%
%% 
\thispagestyle{empty}
\begin{center}
\par \null
\begin{figure}[H]
\centering \includegraphics{unesp.pdf}
\end{figure} 
%\fontfamily{arial}\selectfont
\vspace{3cm}
\fontsize{14}{\baselineskip} \selectfont
{Fulano de Tal} \\  
\vspace{4.5cm}
\onehalfspacing
\fontsize{14}{\baselineskip} \selectfont
{T\'{i}tulo do trabalho} \\ \onehalfspacing \fontsize{14}{\baselineskip} \selectfont
{continua\c{c}\~{a}o do t\'{i}tulo} \\
\vspace{7.0cm}
\fontsize{14}{\baselineskip} \selectfont  
{S\~ao Jos\'e do Rio Preto}\\ \vspace{1.0pt} 
{2018} 
\end{center}
%%
%% **********ROSTO 
%%
\newpage
\thispagestyle{empty}
\setcounter{page}{1}
\begin{center}
\vspace{4cm}
\fontsize{14}{\baselineskip} \selectfont
\vspace{30.0pt}
{Fulano de Tal} \\ \vspace{30.0pt}
{t\'{i}tulo do trabalho} \\ \onehalfspacing \fontsize{14}{\baselineskip} \selectfont
{continua\c{c}\~{a}o do t\'{i}tulo} \\
\par \null
\begin{flushright}
\parbox{3.50in}{
\fontsize{12}{\baselineskip} \selectfont \onehalfspacing
Disserta\c c\~ao apresentada como parte dos requisitos para obten\c c\~ao do t\'itulo de Mestre em Ci\^encia da Computa\c c\~ao, junto ao Programa de P\'os-Gradua\c c\~ao em Ci\^encia da Computa\c c\~ao, do Instituto de Bioci\^encias, Letras e Ci\^encias Exatas da Universidade Estadual Paulista ``J\'ulio de Mesquita Filho'', Campus de S\~ao Jos\'e do Rio Preto. \\ \vspace{1.0pt}
{Orientador: Prof Dr Rodrigo Capobianco Guido } \\ \vspace{1.0pt}
}
\end{flushright}
\fontsize{14}{\baselineskip} \selectfont
\vspace{8.0cm}
S\~ao Jos\'e do Rio Preto, SP \\ \vspace{1.0pt}  
2018
\end{center}
\newpage
\thispagestyle{empty}
%%
%% **********APROVACAO 
%%
\begin{center}
\vspace{4cm}
\fontsize{14}{\baselineskip} \selectfont
\vspace{30.0pt}
{Fulano de Tal} \\ \vspace{30.0pt}
{t\'{i}tulo do trabalho} \\ \onehalfspacing \fontsize{14}{\baselineskip} \selectfont
{continua\c{c}\~{a}o do t\'{i}tulo} \\
\par \null
\begin{flushright}
\parbox{3.50in}{
\fontsize{12}{\baselineskip} \selectfont \onehalfspacing
Disserta\c c\~ao apresentada como parte dos requisitos para obten\c c\~ao do t\'itulo de Mestre em Ci\^encia da Computa\c c\~ao, junto ao Programa de P\'os-Gradua\c c\~ao em Ci\^encia da Computa\c c\~ao, do Instituto de Bioci\^encias, Letras e Ci\^encias Exatas da Universidade Estadual Paulista ``J\'ulio de Mesquita Filho'', Campus de S\~ao Jos\'e do Rio Preto. \\ \vspace{1.0pt}
{Orientador: Prof Dr Rodrigo Capobianco Guido} \\ \vspace{1.0pt}
}
\end{flushright}
\fontsize{14}{\baselineskip} \selectfont
Comiss\~ao Examinadora \\  \vspace{1.0pt}
\end{center}
\fontsize{14}{\baselineskip} \selectfont
Prof Dr Rodrigo Capobianco Guido \\ 
UNESP -  Campus de S\~ao Jos\'e do Rio Preto \\
Co-Orientador \\\\
Prof Dr xxxxxxxx \\ 
UNESP -  Campus de S\~ao Jos\'e do Rio Preto \\\\
Prof Dr xxxxx \\
xxxxxxxx \\
\vspace{3.0cm}
\begin{center}
S\~ao Jos\'e do Rio Preto, SP  \\ \vspace{1.0pt}
2018
\end{center}
%%
%% ********** Resumo
%%
\setlength{\parindent}{0pt}
\newpage \thispagestyle{empty}
\vspace{1.5cm}
\fontsize{12}{\baselineskip} \selectfont
\begin{center}
{\huge{\textbf{AGRADECIMENTOS}}}
\end{center}
\begin{myenv}{1.5}
\vspace{1.5pt}
\vspace{1.5pt}
A minha Mãe e ao determinismo do universo que me trouxe até aqui.
\end{myenv}
%% Resumo/Abstract
%%
%% ********** Resumo
%%
\setlength{\parindent}{0pt}
\newpage \thispagestyle{empty}
\vspace{1.5cm}
\fontsize{12}{\baselineskip} \selectfont
\begin{center}
{\huge{\textbf{RESUMO}}}
\end{center}
\begin{myenv}{1.5}
\fontsize{12}{\baselineskip} \selectfont \onehalfspacing
\par \null
\par \null
1 par\'{a}grafo grande contendo o resumo
\\
\\
Palavras-chave: processamento de sinais, xxxx, xxxx, xxxx
\end{myenv}
%%
%% ********** Resumo
%%
\setlength{\parindent}{0pt}
\newpage \thispagestyle{empty}
\vspace{1.5cm}
\fontsize{12}{\baselineskip} \selectfont
\begin{center}
{\huge{\textit{ \textbf{ABSTRACT}}}}
\end{center}
\begin{myenv}{1.5}
\fontsize{12}{\baselineskip} \selectfont \onehalfspacing
\par \null
\par \null
\textit{resumo em Ingl\^{e}s}\\
\\
\textit{Keywords: signal processing, xxx, xxx, xxx}
\end{myenv}
%% Lista de figuras (gerada automaticamente)
\cleardoublepage
\pagenumbering{gobble}
\listoffigures
% Lista de tabelas (gerada automaticamente)
\cleardoublepage
\pagenumbering{gobble}
\listoftables
\frontmatter
%% Lista de conte\'{u}do (sum\'{a}rio)
\def\contentsname{Sum{\'a}rio} 
\pagenumbering{gobble}
\tableofcontents
\cleardoublepage
% \nobibintoc para bibliografia nao aparecer no indice
%% Gloss\'{a}rio (gerado automaticamente - veja entradas em cap1.tex)
%\cleardoublepage
%\renewcommand{\nomname}{Gloss{\'a}rio}
%\markboth{GLOSS{\'A}RIO}{GLOSS{\'A}RIO}
%\addcontentsline{toc}{chapter}{\nomname}
%\printnomenclature
\mainmatter
\setlength{\parindent}{1.25cm}
%%%%%%%%%%%%%%%%%%%%%%%%%%%%%%%%%%%%%%%%%%%%%%%%%%%%%%%%%%%%%%%%%%%%%%%%%
%%%%%%%%%%%%%%%%%%%%%%%%%%%%%%%%%%%%%%%%%%%%%%%%%%%%%%%%%%%%%%%%%%%%%%%%%
\chapter{Introdu\c c\~ao}
\begin{myenv}{1.5}
\setcounter{page}{12}
\section{Considera\c{c}\~{o}es Iniciais}
Os avan\c cos tecnol\'ogicos  .....
\section{Objetivos}
Este trabalho tem por objetivo ...
\section{Estrutura do trabalho}
O restante deste trabalho est\'{a} organizado da seguinte forma. No Cap\'itulo \ref{c_rl}, xxxxxx. Prosseguindo, o Cap\'itulo \ref{c_ap} cont\'{e}m xxxxx. Em seguida, os testes e resultados constam no Cap\'itulo \ref{c_tr}. Finalmente, o Cap\'itulo \ref{c_c} cont\'{e}m as conclus\~oes, as quais  antecedem as refer\^{e}ncias.
\end{myenv}
%%%%%%%%%%%%%%%%%%%%%%%%%%%%%%%%%%%%%%%%%%%%%%%%%%%%%%%%%%%%%%%%%%%%%%%%%
%%%%%%%%%%%%%%%%%%%%%%%%%%%%%%%%%%%%%%%%%%%%%%%%%%%%%%%%%%%%%%%%%%%%%%%%%
\chapter{Revis\~{a}o Bibliogr\'{a}fica}
\label{c_rl}
\begin{myenv}{1.5}
\textit{Neste Cap\'{i}tulo....}
\section{XXXXX}
\par Os sinais.... de acordo com a Figura \ref{sinaldigital}, observa-se que....
\begin{figure}
\centering 
\begin{tikzpicture} 
\begin{axis}[height=3cm,width=9cm,xlabel=tempo,ylabel=amplitude, xmin=-1,xmax=19, xtick={0,9,18}, ymin=-0.1,ymax=1.1,ytick={0,1}]
\addplot[gray,dotted,mark=none,domain=-1:19,samples=2]{0};
\addplot[brown] coordinates{
(0,0.08208)(1,0.10316)(2,0.14718)(3,0.21919)(4,0.26128)(5,0.31235)(6,0.36336)(7,0.42241)(8,0.48649)(9,0.53854)(10,0.58855)(11,0.63360)(12,0.69369)(13,0.73272)(14,0.77173)(15,0.81879)(16,0.85689)(17,0.88891)(18,0.91892)
};
\addplot[green] coordinates{
(0,0.02202)(1,0.07007)(2,0.10210)(3,0.16016)(4,0.17117)(5,0.17818)(6,0.18418)(7,0.23323)(8,0.28128)(9,0.34735)(10,0.36036)(11,0.36737)(12,0.44244)(13,0.52953)(14,0.54555)(15,0.60160)(16,0.71471)(17,0.73473)(18,0.89890)
};
\end{axis}
\end{tikzpicture}
\caption{representa\c{c}\~{a}o de dois sinais digitais.}
\label{sinaldigital}
\end{figure}
\par Assim sendo.... de acordo com o artigo \cite{mllp}, observa-se que....
\end{myenv}
%%%%%%%%%%%%%%%%%%%%%%%%%%%%%%%%%%%%%%%%%%%%%%%%%%%%%%%%%%%%%%%%%%%%%%%%%
%%%%%%%%%%%%%%%%%%%%%%%%%%%%%%%%%%%%%%%%%%%%%%%%%%%%%%%%%%%%%%%%%%%%%%%%%
\chapter{A Abordagem Proposta}
\label{c_ap}
\textit{Neste Cap\'{i}tulo....}
\begin{myenv}{1.5}
\par bla bla bla....  No artigo \cite{xyz}, nota-se que....
\\
\par bla bla ....
\\
\par bla...
\end{myenv}
%%%%%%%%%%%%%%%%%%%%%%%%%%%%%%%%%%%%%%%%%%%%%%%%%%%%%%%%%%%%%%%%%%%%%%%%%
%%%%%%%%%%%%%%%%%%%%%%%%%%%%%%%%%%%%%%%%%%%%%%%%%%%%%%%%%%%%%%%%%%%%%%%%%
\chapter{Testes e Resultados}
\label{c_tr}
\textit{Neste Cap\'{i}tulo....}
\begin{myenv}{1.5}
\par bla bla bla....
\\
\par bla bla ....  de acordo com a Tabela \ref{t_res}.
\\
\par bla... 
\begin{table}
\centering
\caption{os resultados}
\vspace*{+10pt}
\begin{tabular}{|c|c|c|}
\hline
a & b & c \\
\hline
d & e & f \\
\hline
\end{tabular}
\label{t_res}
\end{table}
\par Assim, nota-se que... 
\end{myenv}
%%%%%%%%%%%%%%%%%%%%%%%%%%%%%%%%%%%%%%%%%%%%%%%%%%%%%%%%%%%%%%%%%%%%%%%%%
%%%%%%%%%%%%%%%%%%%%%%%%%%%%%%%%%%%%%%%%%%%%%%%%%%%%%%%%%%%%%%%%%%%%%%%%%
\chapter{Conclus\~{o}es}
\label{c_c}
\begin{myenv}{1.5}
\par bla bla bla....
\\
\par bla bla ....
\\
\par bla...


    \begin{tabular}{l|c}%
    \bfseries Sexo & \bfseries Idade & \bfseries Arquivo % specify table head
    \csvreader[head to column names]{levantamento.csv}{}% use head of csv as column names
    {\\\hline\Sexo & \Idade & \Arquivo }% specify your coloumns here
    \end{tabular}



\end{myenv}
%%%%%%%%%%%%%%%%%%%%%%%%%%%%%%%%%%%%%%%%%%%%%%%%%%%%%%%%%%%%%%%%%%%%%%%%%
%%%%%%%%%%%%%%%%%%%%%%%%%%%%%%%%%%%%%%%%%%%%%%%%%%%%%%%%%%%%%%%%%%%%%%%%%
\begin{thebibliography}{1}
\bibitem{mllp}Zarate, J.M., Tian, X., Woods, K.J.P. \& Poeppel, D. Multiple levels of linguistic and paralinguistic features contribute to voice recognition. \textit{Sci. Rep.} \textbf{5}, 11475 (2015).
\bibitem{xyz}xxx yyyy zzzz
\end{thebibliography}

\end{document}

