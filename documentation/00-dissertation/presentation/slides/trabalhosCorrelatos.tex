\begin{frame}[allowframebreaks]
	\frametitle{Trabalhos}
	\begin{itemize}
		\item \cite{Ren2019}: Energia e outras várias características do espectro do sinal, SVM.
		\item \cite{DiqunYan2019} \textit{Modelo oculto de Markov} (HMM); \textit{Wavelets}, SVM.
		\item \cite{7802552} \textit{Wavelets}, coeficientes cepstrais (SCCs), Modelos de mistura Gaussiana (GMM).
		\item \cite{alluri2019replay} \textit{"Zero time windowing"}(ZTW), análise cepstral do espectro GMM.
		\item \cite{8725688} Coeficientes cepstrais, GMM.
		\item \cite{Hanilci2018} Predição linear, coeficientes cepstrais, GMM.
		\item \cite{ISI:000473343500086} ``Texturas de voz'', padrões binários locais (LBP) e seus respectivos histogramas, SVM.
		\item Uma abordagem que combina análise de sinal de fala usando a \textit{Transformada Constante Q} (CQT) com o processamento cepstral foi mostrada no artigo \cite{TODISCO2017516}.
		\item No artigo \cite{Patel2015} é usada a \textit{Transformação Auditiva (TU)} que tem como base a transformada \textit{wavelet}, e a \textit{Cochlear Filter Cepstral Coefficients (CFCC)} juntamente com  \textit{estimação da frequência instantânea (IF)}.
		\item \cite{ISI:000490497200068} Partes não vozeadas da fala, três GMMs.
		\item \cite{ISI:000465363900136} amplitude instantânea vinda de flutuações de energia, GMM.
		\item \cite{ISI:000465363900139} Diferenças entre bandas de frequências específicas, \textit{predição linear em domínio de frequência}(FDLP), GMMs.
		\item \cite{Suthokumar2018} \textit{Modulation  spectral  centroid  frequency}, \textit{long term spectral average}, GMM.
		\item \cite{ISI:000458728700054} Envelopamento das amplitudes e das frequências instantâneas em cada banda estreita filtrada, GMM.
		\item \cite{ISI:000392503100008} \textit{Gammatone frequency cepstral coefficients}(MGFCC), GMM.
		\item \cite{8396208} \textit{Hashing} sensível a locus(LSH), MFCC e LSH, tabela \textit{hash}. 
	\end{itemize}
\end{frame}
\begin{frame}
	\frametitle{Comparativo}
		\begin{columns}
			\column{0.5\textwidth}
				\par \textbf{Referências}
				\begin{itemize}
					\item Apenas escala MEL.
					\item Classificadores simples.
					\item Uso escasso de wavelets.
					\item Uso do EER como métrica.
					\item Sem metodologia de seleção automática para os geradores de vetores de características.
				\end{itemize}
			\column{0.5\textwidth}
				\par \textbf{Dissertação}
				\begin{itemize}
					\item Escalas MEL e BARK.
					\item Classificadores simples.
					\item Uso intensivo wavelets.
					\item Uso de EER e acurácia como métrica.
					\item Seleção automática baseada na engenharia paraconsistente de características.
				\end{itemize}
		\end{columns}
\end{frame}