\begin{frame}
	\frametitle{Conclusões e Trabalhos Futuros}
	
	\only<1>{
		\framesubtitle{Conclusões}
		\begin{itemize}
			\item A engenharia paraconsistente de características foi de grande ajuda.
			\item O tipo de \textit{wavelet} utilizada na decomposição dos sinais influencia na criação dos vetores de características.
			\item A combinação \textit{wavelet-packet Haar + escala Bark} foi a melhor.
			\item A combinação \textit{wavelet-packet daubechies 8 + escala Mel} foi a pior.
			\item A escala \textit{Bark} é superior a \textit{MEL} por promover um espalhamento horizontal maior dos valores para os vetores de características.
		\end{itemize}
	}

	\only<2>{
		\framesubtitle{Trabalhos futuros}
		\begin{itemize}
			\item Explorar os resultados das transformadas \textit{wavelets} até um certo nível somente, em oposição a transformações sucessivas.
			\item Separar os componentes de alta e baixa frequência do sinal aparenta ser uma boa base para verificação de \textit{voice spoofing} devido a natureza ruidosa do sinal deste ataque.
			\item Focar a análise em bandas de frequência superiores.
		\end{itemize}
	}
\end{frame}