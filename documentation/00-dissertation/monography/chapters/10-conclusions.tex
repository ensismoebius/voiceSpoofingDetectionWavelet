\chapter{Conclusões e Trabalhos Futuros}
\label{chap:conclusions}
	\par As hipóteses iniciais deste trabalho consideravam que, a decomposição dos sinais de voz no máximo nível possível, usando \textit{wavelets} de qualquer natureza, não alteraria o resultado quando do cálculo de energia do sinal segundo as escalas \textit{BARK} ou \textit{MEL} já que, no final, os intervalos das frequências seriam, de qualquer forma, tratados segundo as regras de cada escala. Após tal fato mostrar-se falso, como constatado nos resultados, que demonstraram ser a combinação \textbf{\textit{wavelet-packet Haar + escala Bark}} a melhor posicionada no plano paraconsistente e \textit{wavelet-packet daubechies 54 + escala Mel} a pior, uma outra hipótese se formou.
	
	\par A segunda hipótese formulada era que as combinações \textit{wavelet + Mel} seriam superiores a \textit{wavelet + Bark}, pois a primeira é otimizada para voz. Novamente, esse pensamento se provou incorreto. Analisando os resultados, é possível notar que a escala \textbf{\textit{Bark} é superior a \textit{MEL} por considerar as frequências mais altas do espectro, inerentes aos processos de regravações}, o que faz com que as diferenças das energias calculadas dentro da mesma classe, genuína ou regravada, seja menor, fazendo com que os vetores de características intraclasses variem menos.

    \par De modo geral, os experimentos mostraram que, \textbf{com ajuda da engenharia paraconsistente de características é possível utilizar classificadores relativamente modestos quando os vetores de características estiverem representando mais propriamente as classes}.
    
    XXXXXX XXXXXX ANDRE COMENTE AQUI SOBRE AS PRINCIPAIS DIFICULDADES, SOBRE O SEU CRESCIMENTO PESSOAL COM O MESTRADO, ETC... E SOBRE IDÉIAS FUTURAS PARA OUTROS ALUNOS SEGUIREM NESTE PROJETO NO FUTURO... 
    
    LEMBRE QUE MUITAS OUTRAS REFERENCIAS DEVEM SER INCLUSAS TAMBEM, CONFORME COMENTEI AO LONGO DOS CAPITULOS ANTERIORES