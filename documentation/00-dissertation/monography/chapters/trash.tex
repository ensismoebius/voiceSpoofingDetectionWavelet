			\subsubsection{Base de comparação}
				\par Para fins comparativos, foi criado um sinal periódico sobre o qual serão aplicadas as transformadas \textit{wavelet} correspondentes aos melhores e piores desempenhos, os resultados dessa aplicação, assim como o sinal original, constituem um gráfico comparativo.
				\par O sinal periódico foi construído com a sequência \textit{32, 10, 20, 38, 37, 28, 38, 34, 18, 24, 24, 9, 23, 24, 28, 34} repetida 32 vezes totalizando um total de 512 posições. Considerando a abordagem de decomposição máxima, foram aplicadas as transformadas \textit{wavelet packet} até o nível 8.
			\subsubsection{Comparativo}
				\par O comparativo foi feito em um gráfico gerado a partir das transformadas \textit{wavelet packet} supracitadas no qual se pode notar as diferenças entre os sinais gerados.
			\subsubsection{Algoritmo}
				\begin{lstlisting}[language=C++, caption={Algoritmo que caracteriza o procedimento 04}, label={lst:experiment04Algo}]
sinal = { 32, 10, 20, 38, 37, 28, 38, 34, 18, 24, 24, 9, 23, 24, 28, 34, 32, 10, 20, 38, 37, 28, 38, 34, 18, 24, 24, 9, 23,...};

listaDeWavelets = { "haar", "daub42", "daub54" };

resultados = {};

for(wavelet : listaDeWavelets){
	sinalTransformado = wavelet(sinal, wavelet);
	
	// Eliminado os dois primeiros sinais 
	// para evitar os picos iniciais que
	// prejudicam a visualizacao do grafico
	sinalTransformado[0] = 0;
	sinalTransformado[1] = 0;
	
	resultados.adicionar(sinalTransformado)
}

gerarGrafico(resultados);
\end{lstlisting}

		\subsection{Experimento 05 - Comparação da escalas BARK e MEL}
		\label{chap:propApproach:sec:Experimento05}
			\subsubsection{Apresentação}
				\par Por fim, nesta parte foram verificados os desempenhos das escalas combinadas com suas respectivas \textit{wavelets}, novamente tais dados foram retirados do experimento \ref{chap:propApproach:sec:Experimento01}.
				\par A geração dos vetores de características foi feita utilizando-se a base de dados já constituída e os mesmos algoritmos de análise com uma pequena diferença: \textbf{A derivação} dos sinais, realizada na finalização da escala \textbf{MEL}, foi \textbf{desativada} para que fosse possível analisar os valores gerados dentro dos seus intervalos que, de outro modo, seriam diminuídos em dois.

				\par O objetivo aqui é mostrar por quê a escala \textbf{BARK} fornece um conjunto melhor de vetores de características dentro do contexto da análise de \textit{voice spoofing}.

			\subsubsection{Comparativo}
				\par O comparativo foi feito em vários gráficos de dispersão que aglutinam os intervalos das escalas em eixos verticais. 
				\par As combinações \textit{wavelet packet + escalas BARK ou MEL} e sinais \textit{spoofing}/\textit{não spoofing} geram, cada uma, um gráfico.
