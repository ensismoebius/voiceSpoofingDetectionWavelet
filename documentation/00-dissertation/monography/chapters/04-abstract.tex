\begin{resumo}
	\par \textit{Voice spoofing} é uma estratégia desenvolvida para ludibriar sistemas de segurança baseados em identificação por voz. Este trabalho visa, para os ataques do tipo \textit{playback speech}, por meio da decomposição do sinal com \textit{wavelets} e posterior cálculo da energia em intervalos baseados na escala BARK ou MEL, determinar qual a melhor combinação BARK/MEL-wavelet para que se obtenha uma separação máxima entre duas classes: \textit{spoofed speech} e locução genuína. Após a apuração da melhor combinação de descritores, por meio da Análise Paraconsistente, os vetores de características oriundos dos sinais de voz são submetidos a ensaios de classificação, variando-se o tamanho do conjunto de treinamento e testes, de forma que, para cada novo teste, esses conjuntos são misturados de forma aleatória. Utilizando as distâncias Euclidiana e Manhattan, além de Máquinas de Vetores de Suporte (SVM), a acurácia máxima obtida foi de 95,3659\% para uma base com 820 sinais.\newline\newline
	Palavras-chave:\ptBRkeyWords.
\end{resumo}

\begin{resumo}[Abstract]
	\begin{otherlanguage*}{english}
		\par Voice spoofing is a strategy designed to circumvent security systems based on voice identification. This work aims, based on speech signals decomposition with wavelets for subsequent BARK or MEL scales energy computations, at determining the best filters and structures to optimally separate between two classes: spoofed and genuine speech, particularly considering playback speech attacks. Once the best combination is obtained, based on Paraconsistent Engineering, the feature vectors are subjected to classification, varying the ramdonly-chosen training and test sets in size. Euclidean and Manhattan distances, as well as Support Vector Machine (SVM), were used as classifiers, where the highest value of accuracy was 95,3659\% for a dataset with 820 signals.\newline\newline
		Key-words:\enUSkeyWords .
	 \end{otherlanguage*}
\end{resumo}






