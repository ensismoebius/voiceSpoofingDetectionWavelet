\chapter{Testes e Resultados} \label{chap:testsResults}
	\begin{myenv}{1.5}
		\section{Experimento 01}
		\begin{figure}[h]
			\centering
			\includegraphics[width=\linewidth]{images/results/paraconsistentPlane/ParaconsistentParcial.png}
			\caption{Distância da combinação Wavelet\textit{X}BARK ou MEL do ponto (1,0)}
			\label{fig:ParaconsistentParcial}
		\end{figure}

		\begin{table}[h]
	\newcommand{\mc}[3]{\multicolumn{#1}{#2}{#3}}
	\definecolor{tcB}{rgb}{0.447059,0.74902,0.266667}
	\definecolor{tcA}{rgb}{0.65098,0.65098,0.65098}
	\definecolor{tcC}{rgb}{1,0.94902,0}
	\begin{center}
		\begin{tabular}{|c|c|c|c|}\hline
			% use packages: color,colortbl
			\rowcolor{tcA}
			Wavelet & G1 & G2 & Distancia do ponto (1,0)\\\hline
			\rowcolor{tcB}
			haar & 0.93615 & 4.68316e-310 & 0.0638503\\\hline
			\mc{1}{|>{\columncolor{tcC}}c|}{daub4} & \mc{1}{>{\columncolor{tcC}}c|}{0.928088} & \mc{1}{>{\columncolor{tcC}}c|}{4.68316e-310} & \mc{1}{>{\columncolor{tcC}}c|}{0.0719123}\\\hline
			\mc{1}{|>{\columncolor{tcC}}c|}{daub6} & \mc{1}{>{\columncolor{tcC}}c|}{0.927885} & \mc{1}{>{\columncolor{tcC}}c|}{4.68316e-310} & \mc{1}{>{\columncolor{tcC}}c|}{0.072115}\\\hline
			\mc{1}{|>{\columncolor{tcC}}c|}{coif6} & \mc{1}{>{\columncolor{tcC}}c|}{0.927823} & \mc{1}{>{\columncolor{tcC}}c|}{4.68316e-310} & \mc{1}{>{\columncolor{tcC}}c|}{0.072177}\\\hline
			\mc{1}{|>{\columncolor{tcC}}c|}{sym8} & \mc{1}{>{\columncolor{tcC}}c|}{0.92769} & \mc{1}{>{\columncolor{tcC}}c|}{4.68316e-310} & \mc{1}{>{\columncolor{tcC}}c|}{0.0723096}\\\hline
			\mc{1}{|>{\columncolor{tcC}}c|}{daub12} & \mc{1}{>{\columncolor{tcC}}c|}{0.926541} & \mc{1}{>{\columncolor{tcC}}c|}{4.68316e-310} & \mc{1}{>{\columncolor{tcC}}c|}{0.073459}\\\hline
		\end{tabular}
	\end{center}
	\caption{Wavelet\textit{X}BARK no plano paraconsistente}
	\label{tab:distParacomBest}
\end{table}

		\par Na figura \ref{fig:ParaconsistentParcial} quanto menor o valor, mais disjuntos tendem ser os vetores de características gerados para as duas classes testadas (\textit{spoofing }e não \textit{spoofing}).

		\par Como se pode constatar, das Wavelets testadas, a que conseguiu \textbf{melhor desempenho} na criação dos melhores vetores de características foi a wavelet \textit{\textbf{Haar com BARK}}. A tabela \ref{tab:distParacomBest} mostra os 6 melhores resultados em termos de distância do ponto (1,0)(verdade) no plano paraconsistente, a totalidade dos dados se encontram no apêndice deste documento nas tabelas \ref{tab:distParacomFrom10Bark_1}, \ref{tab:distParacomFrom10Bark_2}, \ref{tab:distParacomFrom10Mel_1} e \ref{tab:distParacomFrom10Mel_2} e na figura \ref{fig:paraconsistentfull}.
		
		\par A combinação \textbf{\textit{wavelet + BARK}} teve, consistentemente, um desempenho \textbf{melhor} do que as respectivas combinações \textbf{\textit{wavelet + MEL}}.
		
		\par 
		
		\section{Experimento 02}
		
		\begin{figure}[h]
			\centering
			\includegraphics{images/results/confusionMatrices/classifier_Euclidian_10}
			\caption{Acurácia \textit{X} quantidade de testes - Distância Euclidiana, modelo a 10\%}
			\label{fig:classifiereuclidian10}
		\end{figure}
		\begin{table}[h] 					\newcommand{\mc}[3]{\multicolumn{#1}{#2}{#3}} 					\definecolor{tcB}{rgb}{0.447059,0.74902,0.266667} 					\definecolor{tcC}{rgb}{0,0,0} 					\definecolor{tcD}{rgb}{0,0.5,1} 					\definecolor{tcA}{rgb}{0.65098,0.65098,0.65098} 					\begin{center} 						\subfloat[Melhor matriz de confusão]{ 							\begin{tabular}{ccc} 								\mc{1}{l}{} & \mc{1}{>{\columncolor{tcA}}c}{\textbf{genuíno}} & \mc{1}{>{\columncolor{tcA}}c}{\textbf{falsificado}}\\ 								\mc{1}{>{\columncolor{tcA}}r}{\textbf{genuíno}} & \mc{1}{>{\columncolor{tcB}}c}{\textcolor{tcC}{363}} & \mc{1}{>{\columncolor{tcD}}c}{\textcolor{tcC}{14}}\\ 								\mc{1}{>{\columncolor{tcA}}r}{\textbf{falsificado}} & \mc{1}{>{\columncolor{tcD}}c}{\textcolor{tcC}{6}} & \mc{1}{>{\columncolor{tcB}}c}{\textcolor{tcC}{355}} 							\end{tabular} 							\label{tab:classifier_Euclidian_10_best} 						} 						\qquad 						\subfloat[Pior matriz de confusão]{ 							\begin{tabular}{ccc} 								\mc{1}{l}{} & \mc{1}{>{\columncolor{tcA}}c}{\textbf{genuíno}} & \mc{1}{>{\columncolor{tcA}}c}{\textbf{falsificado}}\\ 								\mc{1}{>{\columncolor{tcA}}r}{\textbf{genuíno}} & \mc{1}{>{\columncolor{tcB}}c}{\textcolor{tcC}{275}} & \mc{1}{>{\columncolor{tcD}}c}{\textcolor{tcC}{10}}\\ 								\mc{1}{>{\columncolor{tcA}}r}{\textbf{falsificado}} & \mc{1}{>{\columncolor{tcD}}c}{\textcolor{tcC}{94}} & \mc{1}{>{\columncolor{tcB}}c}{\textcolor{tcC}{359}} 							\end{tabular} 							\label{tab:classifier_Euclidian_10_worse} 						} 					\end{center} 					\caption{Matrizes de confusão para distância Euclidiana com modelo a 10\%} 				\end{table}
		
		\newpage
		\begin{figure}[h]
			\centering
			\includegraphics{images/results/confusionMatrices/classifier_Euclidian_20}
			\caption{Acurácia \textit{X} quantidade de testes - Distância Euclidiana, modelo a 20\%}
			\label{fig:classifiereuclidian20}
		\end{figure}
		\begin{table}[h]
\newcommand{\mc}[3]{\multicolumn{#1}{#2}{#3}}
\definecolor{tcB}{rgb}{0.447059,0.74902,0.266667}
\definecolor{tcC}{rgb}{0,0,0}
\definecolor{tcD}{rgb}{0,0.5,1}
\definecolor{tcA}{rgb}{0.65098,0.65098,0.65098}
\begin{center}
	\begin{tabular}{ccc}
		% use packages: color,colortbl
		\mc{1}{l}{} & \mc{1}{>{\columncolor{tcA}}c}{\textbf{genuine}} & \mc{1}{>{\columncolor{tcA}}c}{\textbf{spoofed}}\\

		\mc{1}{>{\columncolor{tcA}}r}{\textbf{genuine}} & \mc{1}{>{\columncolor{tcB}}c}{\textcolor{tcC}{308}} & \mc{1}{>{\columncolor{tcD}}c}{\textcolor{tcC}{50}}\\

		\mc{1}{>{\columncolor{tcA}}r}{\textbf{spoofed}} & \mc{1}{>{\columncolor{tcD}}c}{\textcolor{tcC}{20}} & \mc{1}{>{\columncolor{tcB}}c}{\textcolor{tcC}{278}}
	\end{tabular}
	\caption{Best confusion matrix for Euclidian distance classifier at 20\% model}
	\label{tab:classifier_Euclidian_20_best}
\end{center}
\end{table}

\begin{table}[h]
	\newcommand{\mc}[3]{\multicolumn{#1}{#2}{#3}}
	\definecolor{tcB}{rgb}{0.447059,0.74902,0.266667}
	\definecolor{tcC}{rgb}{0,0,0}
	\definecolor{tcD}{rgb}{0,0.5,1}
	\definecolor{tcA}{rgb}{0.65098,0.65098,0.65098}
	\begin{center}
		\begin{tabular}{ccc}
			% use packages: color,colortbl
			\mc{1}{l}{} & \mc{1}{>{\columncolor{tcA}}c}{\textbf{genuine}} & \mc{1}{>{\columncolor{tcA}}c}{\textbf{spoofed}}\\
			
			\mc{1}{>{\columncolor{tcA}}r}{\textbf{genuine}} & \mc{1}{>{\columncolor{tcB}}c}{\textcolor{tcC}{295}} & \mc{1}{>{\columncolor{tcD}}c}{\textcolor{tcC}{137}}\\
			
			\mc{1}{>{\columncolor{tcA}}r}{\textbf{spoofed}} & \mc{1}{>{\columncolor{tcD}}c}{\textcolor{tcC}{33}} & \mc{1}{>{\columncolor{tcB}}c}{\textcolor{tcC}{191}}
		\end{tabular}
		\caption{Worst confusion matrix for Euclidian distance classifier at 20\% model}
		\label{tab:classifier_Euclidian_20_worse}
	\end{center}
\end{table}


		\newpage
		\begin{figure}[h]
			\centering
			\includegraphics{images/results/confusionMatrices/classifier_Euclidian_30}
			\caption{Acurácia \textit{X} quantidade de testes - Distância Euclidiana, modelo a 30\%}
			\label{fig:classifiereuclidian30}
		\end{figure}
		\begin{table}[h] 					\newcommand{\mc}[3]{\multicolumn{#1}{#2}{#3}} 					\definecolor{tcB}{rgb}{0.447059,0.74902,0.266667} 					\definecolor{tcC}{rgb}{0,0,0} 					\definecolor{tcD}{rgb}{0,0.5,1} 					\definecolor{tcA}{rgb}{0.65098,0.65098,0.65098} 					\begin{center} 						\subfloat[Best confusion matrix]{ 							\begin{tabular}{ccc} 								\mc{1}{l}{} & \mc{1}{>{\columncolor{tcA}}c}{\textbf{genuine}} & \mc{1}{>{\columncolor{tcA}}c}{\textbf{spoofed}}\\ 								\mc{1}{>{\columncolor{tcA}}r}{\textbf{genuine}} & \mc{1}{>{\columncolor{tcB}}c}{\textcolor{tcC}{283}} & \mc{1}{>{\columncolor{tcD}}c}{\textcolor{tcC}{8}}\\ 								\mc{1}{>{\columncolor{tcA}}r}{\textbf{spoofed}} & \mc{1}{>{\columncolor{tcD}}c}{\textcolor{tcC}{4}} & \mc{1}{>{\columncolor{tcB}}c}{\textcolor{tcC}{279}} 							\end{tabular} 							\label{tab:classifier_Euclidian_30_best} 						} 						\qquad 						\subfloat[Worst confusion matrix]{ 							\begin{tabular}{ccc} 								\mc{1}{l}{} & \mc{1}{>{\columncolor{tcA}}c}{\textbf{genuine}} & \mc{1}{>{\columncolor{tcA}}c}{\textbf{spoofed}}\\ 								\mc{1}{>{\columncolor{tcA}}r}{\textbf{genuine}} & \mc{1}{>{\columncolor{tcB}}c}{\textcolor{tcC}{258}} & \mc{1}{>{\columncolor{tcD}}c}{\textcolor{tcC}{20}}\\ 								\mc{1}{>{\columncolor{tcA}}r}{\textbf{spoofed}} & \mc{1}{>{\columncolor{tcD}}c}{\textcolor{tcC}{29}} & \mc{1}{>{\columncolor{tcB}}c}{\textcolor{tcC}{267}} 							\end{tabular} 							\label{tab:classifier_Euclidian_30_worse} 						} 					\end{center} 					\caption{Confusion matrices for Euclidian distance classifier at 30\% model} 				\end{table}

		\newpage	
		\begin{figure}[h]
			\centering
			\includegraphics{images/results/confusionMatrices/classifier_Euclidian_40}
			\caption{Acurácia \textit{X} quantidade de testes - Distância Euclidiana, modelo a 40\%}
			\label{fig:classifiereuclidian40}
		\end{figure}
		\begin{table}[h]
	\newcommand{\mc}[3]{\multicolumn{#1}{#2}{#3}}
	\definecolor{tcB}{rgb}{0.447059,0.74902,0.266667}
	\definecolor{tcC}{rgb}{0,0,0}
	\definecolor{tcD}{rgb}{0,0.5,1}
	\definecolor{tcA}{rgb}{0.65098,0.65098,0.65098}
	\begin{center}
		\subfloat[Melhor matriz]{
			\begin{tabular}{ccc}
				% use packages: color,colortbl
				\mc{1}{l}{} & \mc{1}{>{\columncolor{tcA}}c}{\textbf{genuíno}} & \mc{1}{>{\columncolor{tcA}}c}{\textbf{falseado}}\\
				
				\mc{1}{>{\columncolor{tcA}}r}{\textbf{genuíno}} & \mc{1}{>{\columncolor{tcB}}c}{\textcolor{tcC}{239}} & \mc{1}{>{\columncolor{tcD}}c}{\textcolor{tcC}{42}}\\
				
				\mc{1}{>{\columncolor{tcA}}r}{\textbf{falseado}} & \mc{1}{>{\columncolor{tcD}}c}{\textcolor{tcC}{7}} & \mc{1}{>{\columncolor{tcB}}c}{\textcolor{tcC}{204}}
			\end{tabular}
			\label{tab:classifier_Euclidian_40_best}
		}
		\qquad
		\subfloat[Pior matriz]{
			\begin{tabular}{ccc}
				% use packages: color,colortbl
				\mc{1}{l}{} & \mc{1}{>{\columncolor{tcA}}c}{\textbf{genuíno}} & \mc{1}{>{\columncolor{tcA}}c}{\textbf{falseado}}\\
				
				\mc{1}{>{\columncolor{tcA}}r}{\textbf{genuíno}} & \mc{1}{>{\columncolor{tcB}}c}{\textcolor{tcC}{232}} & \mc{1}{>{\columncolor{tcD}}c}{\textcolor{tcC}{99}}\\
				
				\mc{1}{>{\columncolor{tcA}}r}{\textbf{falseado}} & \mc{1}{>{\columncolor{tcD}}c}{\textcolor{tcC}{14}} & \mc{1}{>{\columncolor{tcB}}c}{\textcolor{tcC}{147}}
			\end{tabular}
			\label{tab:classifier_Euclidian_40_worse}
		}
	\end{center}
	\caption{Matrizes de confusão para o classificador por distâncias Euclidianas com o uso de 40\% da base para modelagem}
\end{table}

	
		\newpage
		\begin{figure}[h]
			\centering
			\includegraphics{images/results/confusionMatrices/classifier_Euclidian_50}
			\caption{Acurácia \textit{X} quantidade de testes - Distância Euclidiana, modelo a 50\%}
			\label{fig:classifiereuclidian50}
		\end{figure}
		\begin{table}[H] 					\newcommand{\mc}[3]{\multicolumn{#1}{#2}{#3}} 					\definecolor{tcB}{rgb}{0.447059,0.74902,0.266667} 					\definecolor{tcC}{rgb}{0,0,0} 					\definecolor{tcD}{rgb}{0,0.5,1} 					\definecolor{tcA}{rgb}{0.65098,0.65098,0.65098} 					\begin{center} 						\subfloat[Best confusion matrix]{ 							\begin{tabular}{ccc} 								\mc{1}{l}{} & \mc{1}{>{\columncolor{tcA}}c}{\textbf{genuine}} & \mc{1}{>{\columncolor{tcA}}c}{\textbf{spoofed}}\\ 								\mc{1}{>{\columncolor{tcA}}r}{\textbf{genuine}} & \mc{1}{>{\columncolor{tcB}}c}{\textcolor{tcC}{177}} & \mc{1}{>{\columncolor{tcD}}c}{\textcolor{tcC}{19}}\\ 								\mc{1}{>{\columncolor{tcA}}r}{\textbf{spoofed}} & \mc{1}{>{\columncolor{tcD}}c}{\textcolor{tcC}{28}} & \mc{1}{>{\columncolor{tcB}}c}{\textcolor{tcC}{186}} 							\end{tabular} 							\label{tab:classifier_Euclidian_50_best} 						} 						\qquad 						\subfloat[Worst confusion matrix]{ 							\begin{tabular}{ccc} 								\mc{1}{l}{} & \mc{1}{>{\columncolor{tcA}}c}{\textbf{genuine}} & \mc{1}{>{\columncolor{tcA}}c}{\textbf{spoofed}}\\ 								\mc{1}{>{\columncolor{tcA}}r}{\textbf{genuine}} & \mc{1}{>{\columncolor{tcB}}c}{\textcolor{tcC}{149}} & \mc{1}{>{\columncolor{tcD}}c}{\textcolor{tcC}{44}}\\ 								\mc{1}{>{\columncolor{tcA}}r}{\textbf{spoofed}} & \mc{1}{>{\columncolor{tcD}}c}{\textcolor{tcC}{56}} & \mc{1}{>{\columncolor{tcB}}c}{\textcolor{tcC}{161}} 							\end{tabular} 							\label{tab:classifier_Euclidian_50_worse} 						} 					\end{center} 					\caption{Confusion matrices for Euclidian distance classifier at 50\% model} 				\end{table}
	
		\newpage
		\begin{figure}[h]
			\centering
			\includegraphics{images/results/confusionMatrices/classifier_Manhattan_10.png}
			\caption{Acurácia \textit{X} quantidade de testes - Distância Manhattan, modelo a 10\%}
			\label{fig:classifiermanhattan10}
		\end{figure}
		\begin{table}[h] 					\newcommand{\mc}[3]{\multicolumn{#1}{#2}{#3}} 					\definecolor{tcB}{rgb}{0.447059,0.74902,0.266667} 					\definecolor{tcC}{rgb}{0,0,0} 					\definecolor{tcD}{rgb}{0,0.5,1} 					\definecolor{tcA}{rgb}{0.65098,0.65098,0.65098} 					\begin{center} 						\subfloat[Melhor matriz de confusão]{ 							\begin{tabular}{ccc} 								\mc{1}{l}{} & \mc{1}{>{\columncolor{tcA}}c}{\textbf{genuíno}} & \mc{1}{>{\columncolor{tcA}}c}{\textbf{falsificado}}\\ 								\mc{1}{>{\columncolor{tcA}}r}{\textbf{genuíno}} & \mc{1}{>{\columncolor{tcB}}c}{\textcolor{tcC}{365}} & \mc{1}{>{\columncolor{tcD}}c}{\textcolor{tcC}{14}}\\ 								\mc{1}{>{\columncolor{tcA}}r}{\textbf{falsificado}} & \mc{1}{>{\columncolor{tcD}}c}{\textcolor{tcC}{4}} & \mc{1}{>{\columncolor{tcB}}c}{\textcolor{tcC}{355}} 							\end{tabular} 							\label{tab:classifier_Manhattan_10_best} 						} 						\qquad 						\subfloat[Pior matriz de confusão]{ 							\begin{tabular}{ccc} 								\mc{1}{l}{} & \mc{1}{>{\columncolor{tcA}}c}{\textbf{genuíno}} & \mc{1}{>{\columncolor{tcA}}c}{\textbf{falsificado}}\\ 								\mc{1}{>{\columncolor{tcA}}r}{\textbf{genuíno}} & \mc{1}{>{\columncolor{tcB}}c}{\textcolor{tcC}{289}} & \mc{1}{>{\columncolor{tcD}}c}{\textcolor{tcC}{11}}\\ 								\mc{1}{>{\columncolor{tcA}}r}{\textbf{falsificado}} & \mc{1}{>{\columncolor{tcD}}c}{\textcolor{tcC}{80}} & \mc{1}{>{\columncolor{tcB}}c}{\textcolor{tcC}{358}} 							\end{tabular} 							\label{tab:classifier_Manhattan_10_worse} 						} 					\end{center} 					\caption{Matrizes de confusão para distância Manhattan com modelo a 10\%} 				\end{table}
	
		\newpage
		\begin{figure}[h]
			\centering
			\includegraphics{images/results/confusionMatrices/classifier_Manhattan_20.png}
			\caption{Acurácia \textit{X} quantidade de testes - Distância Manhattan, modelo a 20\%}
			\label{fig:classifiermanhattan20}
		\end{figure}
		\begin{table}[h]
\newcommand{\mc}[3]{\multicolumn{#1}{#2}{#3}}
\definecolor{tcB}{rgb}{0.447059,0.74902,0.266667}
\definecolor{tcC}{rgb}{0,0,0}
\definecolor{tcD}{rgb}{0,0.5,1}
\definecolor{tcA}{rgb}{0.65098,0.65098,0.65098}
\begin{center}
	\begin{tabular}{ccc}
		% use packages: color,colortbl
		\mc{1}{l}{} & \mc{1}{>{\columncolor{tcA}}c}{\textbf{Verdadeiro}} & \mc{1}{>{\columncolor{tcA}}c}{\textbf{Falso}}\\

		\mc{1}{>{\columncolor{tcA}}r}{\textbf{Verdadeiro}} & \mc{1}{>{\columncolor{tcB}}c}{\textcolor{tcC}{308}} & \mc{1}{>{\columncolor{tcD}}c}{\textcolor{tcC}{44}}\\

		\mc{1}{>{\columncolor{tcA}}r}{\textbf{Falso}} & \mc{1}{>{\columncolor{tcD}}c}{\textcolor{tcC}{20}} & \mc{1}{>{\columncolor{tcB}}c}{\textcolor{tcC}{284}}
	\end{tabular}
	\caption{Matriz de confusão para o classificador por distâncias Manhattan com o uso de 20\% da base para modelagem}
	\label{tab:classifier_Manhattan_20_best}
\end{center}
\end{table}

\begin{table}[h]
	\newcommand{\mc}[3]{\multicolumn{#1}{#2}{#3}}
	\definecolor{tcB}{rgb}{0.447059,0.74902,0.266667}
	\definecolor{tcC}{rgb}{0,0,0}
	\definecolor{tcD}{rgb}{0,0.5,1}
	\definecolor{tcA}{rgb}{0.65098,0.65098,0.65098}
	\begin{center}
		\begin{tabular}{ccc}
			% use packages: color,colortbl
			\mc{1}{l}{} & \mc{1}{>{\columncolor{tcA}}c}{\textbf{Verdadeiro}} & \mc{1}{>{\columncolor{tcA}}c}{\textbf{Falso}}\\
			
			\mc{1}{>{\columncolor{tcA}}r}{\textbf{Verdadeiro}} & \mc{1}{>{\columncolor{tcB}}c}{\textcolor{tcC}{316}} & \mc{1}{>{\columncolor{tcD}}c}{\textcolor{tcC}{149}}\\
			
			\mc{1}{>{\columncolor{tcA}}r}{\textbf{Falso}} & \mc{1}{>{\columncolor{tcD}}c}{\textcolor{tcC}{12}} & \mc{1}{>{\columncolor{tcB}}c}{\textcolor{tcC}{179}}
		\end{tabular}
		\caption{Pior matriz de confusão para o classificador por distâncias Manhattan com o uso de 20\% da base para modelagem}
		\label{tab:classifier_Manhattan_20_worst}
	\end{center}
\end{table}


		\newpage
		\begin{figure}[h]
			\centering
			\includegraphics{images/results/confusionMatrices/classifier_Manhattan_30.png}
			\caption{Acurácia \textit{X} quantidade de testes - Distância Manhattan, modelo a 30\%}
			\label{fig:classifiermanhattan30}
		\end{figure}
		\begin{table}[h]
	\newcommand{\mc}[3]{\multicolumn{#1}{#2}{#3}}
	\definecolor{tcB}{rgb}{0.447059,0.74902,0.266667}
	\definecolor{tcC}{rgb}{0,0,0}
	\definecolor{tcD}{rgb}{0,0.5,1}
	\definecolor{tcA}{rgb}{0.65098,0.65098,0.65098}
	\begin{center}
		\subfloat[Best matrix]{
			\begin{tabular}{ccc}
				% use packages: color,colortbl
				\mc{1}{l}{} & \mc{1}{>{\columncolor{tcA}}c}{\textbf{genuine}} & \mc{1}{>{\columncolor{tcA}}c}{\textbf{spoofed}}\\
				
				\mc{1}{>{\columncolor{tcA}}r}{\textbf{genuine}} & \mc{1}{>{\columncolor{tcB}}c}{\textcolor{tcC}{276}} & \mc{1}{>{\columncolor{tcD}}c}{\textcolor{tcC}{44}}\\
				
				\mc{1}{>{\columncolor{tcA}}r}{\textbf{spoofed}} & \mc{1}{>{\columncolor{tcD}}c}{\textcolor{tcC}{11}} & \mc{1}{>{\columncolor{tcB}}c}{\textcolor{tcC}{243}}
			\end{tabular}
			\label{tab:classifier_Manhattan_30_best}
		}
		\qquad
		\subfloat[Worst matrix]{
			\begin{tabular}{ccc}
				% use packages: color,colortbl
				\mc{1}{l}{} & \mc{1}{>{\columncolor{tcA}}c}{\textbf{genuine}} & \mc{1}{>{\columncolor{tcA}}c}{\textbf{spoofed}}\\
				
				\mc{1}{>{\columncolor{tcA}}r}{\textbf{genuine}} & \mc{1}{>{\columncolor{tcB}}c}{\textcolor{tcC}{266}} & \mc{1}{>{\columncolor{tcD}}c}{\textcolor{tcC}{106}}\\
				
				\mc{1}{>{\columncolor{tcA}}r}{\textbf{spoofed}} & \mc{1}{>{\columncolor{tcD}}c}{\textcolor{tcC}{21}} & \mc{1}{>{\columncolor{tcB}}c}{\textcolor{tcC}{181}}
			\end{tabular}
			\label{tab:classifier_Manhattan_30_worse}
		}
	\end{center}
	\caption{Confusion matrices for Manhattan distance classifier at 30\% model}
\end{table}
		
		\newpage
		\begin{figure}[h]
			\centering
			\includegraphics{images/results/confusionMatrices/classifier_Manhattan_40.png}
			\caption{Acurácia \textit{X} quantidade de testes - Distância Manhattan, modelo a 40\%}
			\label{fig:classifiermanhattan40}
		\end{figure}
		\begin{table}[h]
	\newcommand{\mc}[3]{\multicolumn{#1}{#2}{#3}}
	\definecolor{tcB}{rgb}{0.447059,0.74902,0.266667}
	\definecolor{tcC}{rgb}{0,0,0}
	\definecolor{tcD}{rgb}{0,0.5,1}
	\definecolor{tcA}{rgb}{0.65098,0.65098,0.65098}
	\begin{center}
		\subfloat[Melhor matriz]{
			\begin{tabular}{ccc}
				% use packages: color,colortbl
				\mc{1}{l}{} & \mc{1}{>{\columncolor{tcA}}c}{\textbf{Verdadeiro}} & \mc{1}{>{\columncolor{tcA}}c}{\textbf{Falso}}\\
				
				\mc{1}{>{\columncolor{tcA}}r}{\textbf{Verdadeiro}} & \mc{1}{>{\columncolor{tcB}}c}{\textcolor{tcC}{239}} & \mc{1}{>{\columncolor{tcD}}c}{\textcolor{tcC}{41}}\\
				
				\mc{1}{>{\columncolor{tcA}}r}{\textbf{Falso}} & \mc{1}{>{\columncolor{tcD}}c}{\textcolor{tcC}{7}} & \mc{1}{>{\columncolor{tcB}}c}{\textcolor{tcC}{205}}
			\end{tabular}
			\label{tab:classifier_Manhattan_40_best}
		}
		\qquad
		\subfloat[Pior matriz]{
			\begin{tabular}{ccc}
				% use packages: color,colortbl
				\mc{1}{l}{} & \mc{1}{>{\columncolor{tcA}}c}{\textbf{Verdadeiro}} & \mc{1}{>{\columncolor{tcA}}c}{\textbf{Falso}}\\
				
				\mc{1}{>{\columncolor{tcA}}r}{\textbf{Verdadeiro}} & \mc{1}{>{\columncolor{tcB}}c}{\textcolor{tcC}{227}} & \mc{1}{>{\columncolor{tcD}}c}{\textcolor{tcC}{91}}\\
				
				\mc{1}{>{\columncolor{tcA}}r}{\textbf{Falso}} & \mc{1}{>{\columncolor{tcD}}c}{\textcolor{tcC}{19}} & \mc{1}{>{\columncolor{tcB}}c}{\textcolor{tcC}{155}}
			\end{tabular}
			\label{tab:classifier_Manhattan_40_worse}
		}
	\end{center}
	\caption{Matrizes de confusão para o classificador por distâncias Manhattan com o uso de 40\% da base para modelagem}
\end{table}

		
		\newpage
		\begin{figure}[h]
			\centering
			\includegraphics{images/results/confusionMatrices/classifier_Manhattan_50.png}
			\caption{Acurácia \textit{X} quantidade de testes - Distância Manhattan, modelo a 50\%}
			\label{fig:classifiermanhattan50}
		\end{figure}
		\begin{table}[H] 					\newcommand{\mc}[3]{\multicolumn{#1}{#2}{#3}} 					\definecolor{tcB}{rgb}{0.447059,0.74902,0.266667} 					\definecolor{tcC}{rgb}{0,0,0} 					\definecolor{tcD}{rgb}{0,0.5,1} 					\definecolor{tcA}{rgb}{0.65098,0.65098,0.65098} 					\begin{center} 						\subfloat[Melhor matriz de confusão]{ 							\begin{tabular}{ccc} 								\mc{1}{l}{} & \mc{1}{>{\columncolor{tcA}}c}{\textbf{genuíno}} & \mc{1}{>{\columncolor{tcA}}c}{\textbf{falseado}}\\ 								\mc{1}{>{\columncolor{tcA}}r}{\textbf{genuíno}} & \mc{1}{>{\columncolor{tcB}}c}{\textcolor{tcC}{203}} & \mc{1}{>{\columncolor{tcD}}c}{\textcolor{tcC}{2}}\\ 								\mc{1}{>{\columncolor{tcA}}r}{\textbf{falseado}} & \mc{1}{>{\columncolor{tcD}}c}{\textcolor{tcC}{2}} & \mc{1}{>{\columncolor{tcB}}c}{\textcolor{tcC}{203}} 							\end{tabular} 							\label{tab:classifier_Manhattan_50_best} 						} 						\qquad 						\subfloat[Pior matriz de confusão]{ 							\begin{tabular}{ccc} 								\mc{1}{l}{} & \mc{1}{>{\columncolor{tcA}}c}{\textbf{genuíno}} & \mc{1}{>{\columncolor{tcA}}c}{\textbf{falseado}}\\ 								\mc{1}{>{\columncolor{tcA}}r}{\textbf{genuíno}} & \mc{1}{>{\columncolor{tcB}}c}{\textcolor{tcC}{188}} & \mc{1}{>{\columncolor{tcD}}c}{\textcolor{tcC}{13}}\\ 								\mc{1}{>{\columncolor{tcA}}r}{\textbf{falseado}} & \mc{1}{>{\columncolor{tcD}}c}{\textcolor{tcC}{17}} & \mc{1}{>{\columncolor{tcB}}c}{\textcolor{tcC}{192}} 							\end{tabular} 							\label{tab:classifier_Manhattan_50_worse} 						} 					\end{center} 					\caption{Matrizes de confusão para distância Manhattan com modelo a 50\%} 				\end{table}

		\section{Experimento 03}

		\newpage
		\begin{figure}[h]
			\centering
			\includegraphics{images/results/confusionMatrices/classifier_SVM_10.png}
			\caption{Acurácia \textit{X} quantidade de testes - SVM, modelo a 10\%}
			\label{fig:classifiersvm10}
		\end{figure}
		\begin{table}[h] 					\newcommand{\mc}[3]{\multicolumn{#1}{#2}{#3}} 					\definecolor{tcB}{rgb}{0.447059,0.74902,0.266667} 					\definecolor{tcC}{rgb}{0,0,0} 					\definecolor{tcD}{rgb}{0,0.5,1} 					\definecolor{tcA}{rgb}{0.65098,0.65098,0.65098} 					\begin{center} 						\subfloat[Melhor matriz de confusão]{ 							\begin{tabular}{ccc} 								\mc{1}{l}{} & \mc{1}{>{\columncolor{tcA}}c}{\textbf{genuíno}} & \mc{1}{>{\columncolor{tcA}}c}{\textbf{falsificado}}\\ 								\mc{1}{>{\columncolor{tcA}}r}{\textbf{genuíno}} & \mc{1}{>{\columncolor{tcB}}c}{\textcolor{tcC}{366}} & \mc{1}{>{\columncolor{tcD}}c}{\textcolor{tcC}{10}}\\ 								\mc{1}{>{\columncolor{tcA}}r}{\textbf{falsificado}} & \mc{1}{>{\columncolor{tcD}}c}{\textcolor{tcC}{3}} & \mc{1}{>{\columncolor{tcB}}c}{\textcolor{tcC}{359}} 							\end{tabular} 							\label{tab:classifier_SVM_10_best} 						} 						\qquad 						\subfloat[Pior matriz de confusão]{ 							\begin{tabular}{ccc} 								\mc{1}{l}{} & \mc{1}{>{\columncolor{tcA}}c}{\textbf{genuíno}} & \mc{1}{>{\columncolor{tcA}}c}{\textbf{falsificado}}\\ 								\mc{1}{>{\columncolor{tcA}}r}{\textbf{genuíno}} & \mc{1}{>{\columncolor{tcB}}c}{\textcolor{tcC}{284}} & \mc{1}{>{\columncolor{tcD}}c}{\textcolor{tcC}{9}}\\ 								\mc{1}{>{\columncolor{tcA}}r}{\textbf{falsificado}} & \mc{1}{>{\columncolor{tcD}}c}{\textcolor{tcC}{85}} & \mc{1}{>{\columncolor{tcB}}c}{\textcolor{tcC}{360}} 							\end{tabular} 							\label{tab:classifier_SVM_10_worse} 						} 					\end{center} 					\caption{Matrizes de confusão para SVM com modelo a 10\%} 				\end{table}

		\newpage
		\begin{figure}[h]
			\centering
			\includegraphics{images/results/confusionMatrices/classifier_SVM_20.png}
			\caption{Acurácia \textit{X} quantidade de testes - SVM, modelo a 20\%}
			\label{fig:classifiersvm20}
		\end{figure}
		\begin{table}[h]
\newcommand{\mc}[3]{\multicolumn{#1}{#2}{#3}}
\definecolor{tcB}{rgb}{0.447059,0.74902,0.266667}
\definecolor{tcC}{rgb}{0,0,0}
\definecolor{tcD}{rgb}{0,0.4,0.701961}
\definecolor{tcA}{rgb}{0.65098,0.65098,0.65098}
\begin{center}
	\begin{tabular}{ccc}
		% use packages: color,colortbl
		\mc{1}{l}{} & \mc{1}{>{\columncolor{tcA}}c}{\textbf{Verdadeiro}} & \mc{1}{>{\columncolor{tcA}}c}{\textbf{Falso}}\\

		\mc{1}{>{\columncolor{tcA}}r}{\textbf{Verdadeiro}} & \mc{1}{>{\columncolor{tcB}}c}{\textcolor{tcC}{301}} & \mc{1}{>{\columncolor{tcD}}c}{\textcolor{tcC}{18}}\\

		\mc{1}{>{\columncolor{tcA}}r}{\textbf{Falso}} & \mc{1}{>{\columncolor{tcD}}c}{\textcolor{tcC}{27}} & \mc{1}{>{\columncolor{tcB}}c}{\textcolor{tcC}{310}}
	\end{tabular}
	\caption{Tabela de confusão para classificador SVM 20\%}
	\label{tab:classifier_SVM_20}
\end{center}
\end{table}

		
		\newpage
		\begin{figure}[h]
			\centering
			\includegraphics{images/results/confusionMatrices/classifier_SVM_30.png}
			\caption{Acurácia \textit{X} quantidade de testes - SVM, modelo a 30\%}
			\label{fig:classifiersvm30}
		\end{figure}
		\begin{table}[h]
	\newcommand{\mc}[3]{\multicolumn{#1}{#2}{#3}}
	\definecolor{tcB}{rgb}{0.447059,0.74902,0.266667}
	\definecolor{tcC}{rgb}{0,0,0}
	\definecolor{tcD}{rgb}{0,0.5,1}
	\definecolor{tcA}{rgb}{0.65098,0.65098,0.65098}
	\begin{center}
		\subfloat[Melhor matriz]{
			\begin{tabular}{ccc}
				% use packages: color,colortbl
				\mc{1}{l}{} & \mc{1}{>{\columncolor{tcA}}c}{\textbf{Verdadeiro}} & \mc{1}{>{\columncolor{tcA}}c}{\textbf{Falso}}\\
				
				\mc{1}{>{\columncolor{tcA}}r}{\textbf{Verdadeiro}} & \mc{1}{>{\columncolor{tcB}}c}{\textcolor{tcC}{268}} & \mc{1}{>{\columncolor{tcD}}c}{\textcolor{tcC}{14}}\\
				
				\mc{1}{>{\columncolor{tcA}}r}{\textbf{Falso}} & \mc{1}{>{\columncolor{tcD}}c}{\textcolor{tcC}{19}} & \mc{1}{>{\columncolor{tcB}}c}{\textcolor{tcC}{273}}
			\end{tabular}
			\label{tab:classifier_SVM_30_best}
		}
		\qquad
		\subfloat[Pior matriz]{
			\begin{tabular}{ccc}
				% use packages: color,colortbl
				\mc{1}{l}{} & \mc{1}{>{\columncolor{tcA}}c}{\textbf{Verdadeiro}} & \mc{1}{>{\columncolor{tcA}}c}{\textbf{Falso}}\\
				
				\mc{1}{>{\columncolor{tcA}}r}{\textbf{Verdadeiro}} & \mc{1}{>{\columncolor{tcB}}c}{\textcolor{tcC}{249}} & \mc{1}{>{\columncolor{tcD}}c}{\textcolor{tcC}{43}}\\
				
				\mc{1}{>{\columncolor{tcA}}r}{\textbf{Falso}} & \mc{1}{>{\columncolor{tcD}}c}{\textcolor{tcC}{38}} & \mc{1}{>{\columncolor{tcB}}c}{\textcolor{tcC}{244}}
			\end{tabular}
			\label{tab:classifier_SVM_30_worse}
		}
	\end{center}
	\caption{Matrizes de confusão para o classificador SVM com o uso de 30\% da base para modelagem}
\end{table}

	
		\newpage
		\begin{figure}[h]
			\centering
			\includegraphics{images/results/confusionMatrices/classifier_SVM_40.png}
			\caption{Acurácia \textit{X} quantidade de testes - SVM, modelo a 40\%}
			\label{fig:classifiersvm40}
		\end{figure}
		\begin{table}[H] 					\newcommand{\mc}[3]{\multicolumn{#1}{#2}{#3}} 					\definecolor{tcB}{rgb}{0.447059,0.74902,0.266667} 					\definecolor{tcC}{rgb}{0,0,0} 					\definecolor{tcD}{rgb}{0,0.5,1} 					\definecolor{tcA}{rgb}{0.65098,0.65098,0.65098} 					\begin{center} 						\subfloat[Best confusion matrix]{ 							\begin{tabular}{ccc} 								\mc{1}{l}{} & \mc{1}{>{\columncolor{tcA}}c}{\textbf{genuine}} & \mc{1}{>{\columncolor{tcA}}c}{\textbf{spoofed}}\\ 								\mc{1}{>{\columncolor{tcA}}r}{\textbf{genuine}} & \mc{1}{>{\columncolor{tcB}}c}{\textcolor{tcC}{232}} & \mc{1}{>{\columncolor{tcD}}c}{\textcolor{tcC}{18}}\\ 								\mc{1}{>{\columncolor{tcA}}r}{\textbf{spoofed}} & \mc{1}{>{\columncolor{tcD}}c}{\textcolor{tcC}{14}} & \mc{1}{>{\columncolor{tcB}}c}{\textcolor{tcC}{228}} 							\end{tabular} 							\label{tab:classifier_SVM_40_best} 						} 						\qquad 						\subfloat[Worst confusion matrix]{ 							\begin{tabular}{ccc} 								\mc{1}{l}{} & \mc{1}{>{\columncolor{tcA}}c}{\textbf{genuine}} & \mc{1}{>{\columncolor{tcA}}c}{\textbf{spoofed}}\\ 								\mc{1}{>{\columncolor{tcA}}r}{\textbf{genuine}} & \mc{1}{>{\columncolor{tcB}}c}{\textcolor{tcC}{207}} & \mc{1}{>{\columncolor{tcD}}c}{\textcolor{tcC}{39}}\\ 								\mc{1}{>{\columncolor{tcA}}r}{\textbf{spoofed}} & \mc{1}{>{\columncolor{tcD}}c}{\textcolor{tcC}{39}} & \mc{1}{>{\columncolor{tcB}}c}{\textcolor{tcC}{207}} 							\end{tabular} 							\label{tab:classifier_SVM_40_worse} 						} 					\end{center} 					\caption{Confusion matrices for SVM classifier at 40\% model} 				\end{table}
	
		\newpage
		\begin{figure}[h]
			\centering
			\includegraphics{images/results/confusionMatrices/classifier_SVM_50.png}
			\caption{Acurácia \textit{X} quantidade de testes - SVM, modelo a 50\%}
			\label{fig:classifiersvm50}
		\end{figure}
		\begin{table}[h]
\newcommand{\mc}[3]{\multicolumn{#1}{#2}{#3}}
\definecolor{tcB}{rgb}{0.447059,0.74902,0.266667}
\definecolor{tcC}{rgb}{0,0,0}
\definecolor{tcD}{rgb}{0,0.4,0.701961}
\definecolor{tcA}{rgb}{0.65098,0.65098,0.65098}
\begin{center}
	\begin{tabular}{ccc}
		% use packages: color,colortbl
		\mc{1}{l}{} & \mc{1}{>{\columncolor{tcA}}c}{\textbf{Verdadeiro}} & \mc{1}{>{\columncolor{tcA}}c}{\textbf{Falso}}\\

		\mc{1}{>{\columncolor{tcA}}r}{\textbf{Verdadeiro}} & \mc{1}{>{\columncolor{tcB}}c}{\textcolor{tcC}{196}} & \mc{1}{>{\columncolor{tcD}}c}{\textcolor{tcC}{10}}\\

		\mc{1}{>{\columncolor{tcA}}r}{\textbf{Falso}} & \mc{1}{>{\columncolor{tcD}}c}{\textcolor{tcC}{9}} & \mc{1}{>{\columncolor{tcB}}c}{\textcolor{tcC}{195}}
	\end{tabular}
	\caption{Melhor tabela de confusão para classificador SVM 50\%}
	\label{tab:classifier_SVM_50_best}
\end{center}
\end{table}

\begin{table}[h]
	\newcommand{\mc}[3]{\multicolumn{#1}{#2}{#3}}
	\definecolor{tcB}{rgb}{0.447059,0.74902,0.266667}
	\definecolor{tcC}{rgb}{0,0,0}
	\definecolor{tcD}{rgb}{0,0.4,0.701961}
	\definecolor{tcA}{rgb}{0.65098,0.65098,0.65098}
	\begin{center}
		\begin{tabular}{ccc}
			% use packages: color,colortbl
			\mc{1}{l}{} & \mc{1}{>{\columncolor{tcA}}c}{\textbf{Verdadeiro}} & \mc{1}{>{\columncolor{tcA}}c}{\textbf{Falso}}\\
			
			\mc{1}{>{\columncolor{tcA}}r}{\textbf{Verdadeiro}} & \mc{1}{>{\columncolor{tcB}}c}{\textcolor{tcC}{175}} & \mc{1}{>{\columncolor{tcD}}c}{\textcolor{tcC}{30}}\\
			
			\mc{1}{>{\columncolor{tcA}}r}{\textbf{Falso}} & \mc{1}{>{\columncolor{tcD}}c}{\textcolor{tcC}{30}} & \mc{1}{>{\columncolor{tcB}}c}{\textcolor{tcC}{175}}
		\end{tabular}
		\caption{Pior tabela de confusão para classificador SVM 50\%}
		\label{tab:classifier_SVM_50_worse}
	\end{center}
\end{table}

		
	\end{myenv}

